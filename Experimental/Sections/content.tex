\section{Introduction}

\begin{frame}{Experimental Physics: A Historical Perspective}
\onslide<1->{In 1923, Felix Auerbach claimed that:}
\onslide<2->{
\begin{quote}
X-rays are not a natural phenomenon... they were invented by R\"ontgen.
\end{quote}}
\onslide<3->{
This provocative view highlights the artificial character of experiments:
\begin{itemize}
  \item Experiments create, not just observe.
  \item Knowledge arises from human-made phenomena.
\end{itemize}}
\end{frame}

\begin{frame}{From Observation to Intervention}
\onslide<1->{Since the 17th century, scholars debated:}
\onslide<2->{
\begin{itemize}
  \item Is experiment a valid path to knowledge?
  \item Can making replace observing?
\end{itemize}}
\onslide<3->{
The experimentalist emerged to bridge theory and practice.
}
\end{frame}

\begin{frame}{Christian Wolff's \textit{Third Man} (1764)}
\onslide<1->{Wolff called for a new figure:}
\onslide<2->{
\begin{quote}
A third man who unites science and art.
\end{quote}}
\onslide<3->{
\begin{itemize}
  \item Rejected by both theorists and artisans.
  \item Compared to a bat: neither bird nor quadruped.
\end{itemize}}
\end{frame}

\begin{frame}{Experimental Physics Enters Academia}
\onslide<1->{18th--19th centuries: Experimentalists faced epistemic tension.}
\onslide<2->{
\begin{itemize}
  \item Is experimental knowledge real science?
  \item Is manual manipulation compatible with scholarly traditions?
\end{itemize}}
\onslide<3->{
Over time, head and hand work gained legitimacy in science.
}
\end{frame}

\section{The Role of Instruments}

\begin{frame}{Instruments as Extensions of the Senses}
\onslide<1->{New fields like electricity relied on instruments.}
\onslide<2->{
\begin{itemize}
  \item Vacuum tubes modeled auroras.
  \item Volta's battery revealed microphysical phenomena.
\end{itemize}}
\onslide<3->{
Devices created phenomena that otherwise could not be observed.
}
\end{frame}

\begin{frame}{Controversies of Artificial Experience}
\onslide<1->{Artificial labs led to debates:}
\onslide<2->{
\begin{itemize}
  \item Is knowledge from artificial settings valid?
  \item What is the scope of lab-based insights?
\end{itemize}}
\onslide<3->{
State and industrial support helped labs enter universities.
}
\end{frame}

\section{From Artisans to Scientists}

\begin{frame}{Merging Traditions: The \textit{Handwerksgelehrte}}
\onslide<1->{Late 19th century: experimentalists joined academia.}
\onslide<2->{
\begin{itemize}
  \item "Scholars of the crafts"
  \item Lab knowledge gained equal status to textual knowledge
\end{itemize}}
\onslide<3->{
This marked the birth of modern experimental science.
}
\end{frame}

\begin{frame}{New Teaching and New Methodologies}
\onslide<1->{Experimental chairs were created in universities.}
\onslide<2->{
\begin{itemize}
  \item Helmholtz and Maxwell promoted sensory experience
  \item Physics teaching emphasized tools and hands-on work
\end{itemize}}
\onslide<3->{
\begin{quote}
Facts must be felt, not learned from description. -- Maxwell
\end{quote}}
\end{frame}

\section{Reflections on Experience and Theory}

\begin{frame}{Philosophical Reflections}
\onslide<1->{Debates persisted over theory vs. experience.}
\onslide<2->{
Joseph Dietzgen (1869):
\begin{quote}
Even the lowest art of experiment is connected to theory.
\end{quote}}
\onslide<3->{
Materialist and idealist views of knowledge needed mediation.
}
\end{frame}

\begin{frame}{The Artificial as Method}
\onslide<1->{By 1900, physics was seen as a technical science.}
\onslide<2->{
\begin{itemize}
  \item It created artificial phenomena.
  \item It required intentional acts.
\end{itemize}}
\onslide<3->{
\textbf{Otto Wiener:} Instruments extend human senses.
}
\end{frame}

\begin{frame}{Condensed Experience}
\onslide<1->{Auerbach: theory builds from experience.}
\onslide<2->{
\begin{itemize}
  \item Not a test of theory, but its foundation.
  \item Like a dynamo: small experience sparks large knowledge.
\end{itemize}}
\onslide<3->{
Abstract structures must always check back with reality.
}
\end{frame}

\begin{frame}{Theoretical Extremes}
\onslide<1->{Some theorists reversed the relation:}
\onslide<2->{
\begin{quote}
If the world contradicts theory, the world must be wrong.
\end{quote}}
\onslide<3->{
Theory and experience became increasingly specialized.
}
\end{frame}

\section{Conclusion}

\begin{frame}{Conclusion: An Epistemological Shift}
\onslide<1->{By the 20th century, experiment became central to science.}
\onslide<2->{
\begin{itemize}
  \item Instruments enabled new phenomena.
  \item Sensuous experience gained epistemic legitimacy.
\end{itemize}}
\onslide<3->{
Experimental physics unified knowing and doing.
}
\end{frame}
