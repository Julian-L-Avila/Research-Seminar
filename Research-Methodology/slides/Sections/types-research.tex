\begin{frame}{Research Classification}
  Research is broadly classified into:
  \begin{enumerate}
    \item \textbf{Fundamental (Basic) Research}
    \item \textbf{Applied Research}
  \end{enumerate}
\end{frame}

\begin{frame}{Basic Research}
  Basic research investigates \textbf{fundamental principles and theories}.
  It seeks to understand why phenomena occur, without immediate application.

  \textbf{Example:}

  The \textbf{Dirac Equation}, a cornerstone for the Standard Model
  in particle physics.
\end{frame}

\begin{frame}{Applied Research}
  Applied research solves practical problems using established theories.

  \textbf{Example:}
  \textbf{Quantum Topological States of Matter}, leading to advances
  in fault-tolerant quantum computing. \cite{Stanescu2017}

  Applied research often drives technological innovation.
\end{frame}

\begin{frame}{Normal vs. Revolutionary Research}
  Research follows two main patterns:
  \begin{itemize}
    \item \textbf{Normal Research}: Works within an accepted paradigm,
      following established rules and methods.
    \item \textbf{Revolutionary Research}: Occurs when unexpected results
      challenge the existing paradigm, leading to a paradigm shift.
  \end{itemize}
\end{frame}

\begin{frame}{Paradigm Shifts}
  A \textbf{scientific revolution} occurs when accumulated anomalies force
  a shift in fundamental theories.

  This leads to a new paradigm where research
  continues under revised principles.
\end{frame}
