\begin{frame}{Stages of Research}
  Scientific problem-solving follows structured stages:
  \begin{enumerate}
    \item Selection of a research topic
    \item Problem definition
    \item Literature review and reference collection
    \item Assessment of current knowledge
    \item Hypothesis formulation
    \item Research design
    \item Data collection
    \item Data analysis
    \item Interpretation of results
    \item Report writing
  \end{enumerate}
\end{frame}

\begin{frame}{Identifying a Research Topic}
  Research topics emerge from:
  \begin{itemize}
    \item Personal interest in a theory or phenomenon
    \item Unresolved challenges in science and technology
    \item Recent trends and unexplored areas
    \item Discussions with experts and mentors
  \end{itemize}
  Reviewing advanced textbooks and recent papers refines the research scope.
\end{frame}

\begin{frame}{Formulating the Problem}
  A well-defined research problem should:
  \begin{itemize}
    \item Be clearly formulated, preferably as a question
    \item Have a precise scope and well-defined assumptions
    \item Consider feasibility in terms of:
  \end{itemize}
  \begin{enumerate}
    \item Scientific significance and originality
    \item Contribution to existing knowledge
    \item Availability of supervision and guidance
    \item Practical completion within time constraints
    \item Access to required resources (equipment, data, literature)
  \end{enumerate}
\end{frame}

\begin{frame}{Literature Review}
  A literature review synthesizes existing knowledge from various sources.

  \textbf{Key sources:}
  \begin{itemize}
    \item Review articles and meta-analyses
    \item Research journals and conference papers
    \item Advanced textbooks and monographs
  \end{itemize}

  \textbf{Benefits:}
  \begin{itemize}
    \item Refining and contextualizing the research problem
    \item Understanding theoretical and practical aspects
    \item Identifying gaps and potential contributions
  \end{itemize}
\end{frame}

\begin{frame}{Reference Collection}
  Organizing references systematically ensures efficiency in future work.

  \textbf{Essential sources for physics research:}
  \begin{itemize}
    \item \textit{Physics Reports}, \textit{Reviews of Modern Physics}
    \item \textit{Physical Review Letters}, \textit{American Journal of Physics}
    \item \textit{Pramana}, \textit{Current Science}
    \item Conference proceedings and technical reports
  \end{itemize}
\end{frame}

