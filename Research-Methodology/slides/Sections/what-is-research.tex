\begin{frame}{Research for the Truth}
  Research is a \textbf{logical} and \textbf{systematic} process of knowledge
  discovery. It seeks to answer \textbf{how} and \textbf{what}, explaining
  relations, predicting events, and forming theories. \cite{Rajasekar2006}
\end{frame}

\begin{frame}{Quote}
  \centering
  \textit{Scientific research is a chaotic business, stumbling along amidst red
    herrings, errors, and truly creative insights. Great scientific breakthroughs
    are rarely the work of a single researcher plodding inexorably toward a goal.
    The crucial idea behind the breakthrough may surface a number of times, in
    different places, only to sink again beneath the babble of an endless
  scientific discourse.} \cite{Rajasekar2006}
\end{frame}

\begin{frame}{Example: The Dirac Equation}
  An example is \textbf{Paul Dirac's  wave equation}
  for a relativistic electron.

  \begin{equation*}
    \left( i \hbar \gamma^{\mu} \partial_{\mu} - mc \right) \psi  = 0
  \end{equation*}

  \vspace{0.5cm}
  
  Previous works from Schrödinger and Klein-Gordon. \cite{Dirac1928}

  It successfully combines \textbf{quantum mechanics}
  and \textbf{special relativity}, predicting the existence of antimatter.
\end{frame}

\begin{frame}{Objectives of Research}
  The main objectives of research are:
  \begin{itemize}
    \item \textbf{Discover} new facts
    \item \textbf{Verify} and test existing facts
    \item \textbf{Identify} causes and effects of events
    \item \textbf{Solve} theoretical and practical problems
  \end{itemize}
\end{frame}

\begin{frame}{Research and Nature}
  Research helps us understand \textbf{nature} and uncover its fundamental laws.

  \begin{equation*}
    \gamma^{0} = \sigma^{3} \otimes I_2 \, , \,
    \gamma^{j} = i \sigma^{2} \otimes \sigma^{j}
  \end{equation*}

  For example, \textbf{the Dirac equation} revealed the intrinsic property of
  \textbf{electron spin}, revolutionizing quantum physics. \cite{mehranshargh2023}
\end{frame}
