In 2002 two highly publicized events shaĴered the common
complacent view that the quantitative nature of physics research
and strong peer- review practices would shelter the discipline
from ethics violations. The fi rst, at Lawrence Berkeley National
Laboratory, was the retraction of Victor Ninov’s claimed discovery
of two new elements1 (atomic numbers 116 and 118). The other, at Lucent
Bell Labs, was mounting suspicions about Jan Hendrik Schön’s data
showing extraordinary properties of many novel materials, including
high-temperature superconductors and thin films for device applications.
(See PѕѦѠіѐѠ TќёюѦ, November 2002, page 15.) Investigations at both
institutions uncovered fl agrant data fabrication. Those events showed
that ethical practice in physics could not be taken for granted and added
to a growing awareness that ethical practice in scientifi c research was
not a given.


he American Physical Society’s Panel
on Public Affairs (POPA), which in 2002
had the primary responsibility for ethical
maĴers, commissioned a task force charged
with understanding how physicists are
taught about ethics and with making rec-
ommendations for further actions APS
could take to address ethical concerns.
The most informative survey was
of what were then called junior members,
which roughly corresponds to today’s APS
Early Career members. Those physicists
had acquired their PhDs three years or less
before the survey and could speak to their
experiences as students, postdocs, and newly
independent researchers.

They were asked
how they learned about ethical practices and what their experiences were with ethics issues in their
research training. That survey found a distressing rate of un-
ethical research practices and a lack of formal ethics training,

n 2020 a follow- up survey was sent out to two APS member
cohorts, early- career scientists and graduate students, to inves-
tigate whether ethics awareness and practice had changed
since the original survey. The data show that although ethics
education improved over the 17 years—addressing what a 2003
respondent called “the silence that exists now”—serious chal-
lenges remain.


BOX 1
The Office of Science and Technology Policy defines “research
misconduct” as “fabrication, falsification, or plagiarism in propos-
ing, performing, or reviewing research, or in reporting research
results.”7
‣ Fabrication is making up data or results and recording or
reporting them.
‣ Falsification is manipulating research materials, equipment,
or processes or changing or omitting data or results such that the
research is not accurately represented in the research record.
‣ Plagiarism is the appropriation of another person’s ideas,
processes, results, or words without giving appropriate credit.
The office’s research misconduct policy also sets the legal
threshold for charges of misconduct. To be considered research
misconduct, actions must represent a “significant departure from
accepted practices,” be “committed intentionally, or knowingly, or
recklessly,” and be “proven by a preponderance of evidence.” 7


BOX 2
The 2003 American Physical Society survey was the first to exam-
ine ethics in practice in physics and among the first to examine
ethics in any of the physical sciences. Since then important sur-
veys of physics and other disciplines have been published and
revealed nuances in how the scientific enterprise works. The fre-
quency of misconduct is somewhat lower in the physical sciences
than in biological, medical, and social sciences, but the patterns
and types of misconduct are similar. Those patterns help pinpoint
where significant improvements in ethics education and practice
are needed in all sciences.
Reports examined ethics in medical physics, 8 sexual harass-
ment experienced by female undergraduate physics majors and
how that negatively affects their persistence in STEM (science,
technology, engineering, and mathematics) fields,9 the scope of
National Institutes of Health–funded scientists’ misconduct be-
yond fabrication, falsification, and plagiarism, 10 and research
practices across disciplines in the Netherlands.11 The Dutch sur-
vey, for example, showed that half the respondents admitted to
questionable research practices and that about 4% said they had
fabricated or falsified data in the preceding three years. The find-
ings in those publications are consistent with those of the Amer-
ican Physical Society surveys in 2020 and 2003.

BOX 3
The American Physical Society Guidelines on Ethics rest on the
principles given in its preamble (https://www.aps.org/policy
/statements/19_1.cfm): “As citizens of the global community of
science, physicists share responsibility for its welfare. The success
of the scientific enterprise rests upon two ethical pillars. The first
of them is the obligation to tell the truth, which includes avoiding
fabrication, falsification, and plagiarism. The second is the obli-
gation to treat people well, which prohibits abuse of power, en-
courages fair and respectful relationships with colleagues,
subordinates, and students, and eschews bias, whether implicit or
explicit. Professional integrity in the conception, conduct, and
communication of physics activities reflects not only on the rep-
utations of individual physicists and their organizations, but also
on the image and credibility of the physics profession in the eyes
of scientific colleagues, government, and the public. Physicists
must adopt high standards of ethical behavior, and transmit im-
proving practices with enthusiasm to future generations.”


he new survey also included questions on harassment that
had not been posed in 2003. To those questions overall, there
were 3577 responses from graduate students and early- career
APS members, of whom 795 identified as women, 2348 identi-
fied as men, 37 identified as neither women nor
men, and 397 preferred not to identify gender.
The differences between the experiences of men
and women are striking, as shown in fi gure 4.
Women are five times as likely as men are to feel
that they were treated differently, ignored, or put
down because of their group affiliation and to
have heard comments of a sexual nature or tone.
Around 15% of the female respondents reported
being touched without permission compared
with 2% of male respondents. The wriĴen com-
ments even included multiple reports of rape by
coworkers. Respondents with gender identities
other than male or female gave responses be-
tween those of men and women

ecommendations
With the concerns of students and early-career physicists
in mind, the APS Ethics CommiĴee formulated a number of
recommendations currently under consideration by APS
leadership.
1. Develop educational materials
Although ethics education has improved at the university level
over the past two decades— driven in part by NSF require-
ments for responsible and ethical conduct of research and Title
IX compliance— the survey results cast doubt on its effect. That
minimal influence may be because of the nature of most formal
institutional ethics training: largely web based, without de-
tailed discussions of situations, and lacking opportunities for
questions. APS should develop new materials that are relevant
to physics and effective.
2. Foster more respectful behavior
Changing the physics culture to embrace respectful treatment
of others as a core value could help reduce instances of harass-
ment, discrimination, and toxic power dynamics.3 Much work
remains to reduce the pressures that have fueled and enabled
such behavior. A new initiative, the APS Inclusion, Diversity,
and Equity Alliance, helps physics departments and laborato-
While in context associated with physics has someone
behaved in the following ways?
Treated you differently, ignored you, or put
you down because of your sex, gender,
race, or other group affiliation.
Made sexist or racist remarks or told
inappropriate jokes or stories that
disparaged groups or people.
Made remarks suggesting people of your
sex, gender, race, ethnicity, or other group
affiliation are not as good at physics.
Made comments of a sexual nature or tone
about your body, appearance, or clothing
or discussed your sexual activity.
Touched you without your permission,
making you uncomfortable.
0 10 20 30 40 50
Survey respondents (%)Female Additional
identities Male
FIGURE 4. HARASSMENT is a common experience, particularly for women, as reported
by graduate students and early-career American Physical Society members in 2020. The
397 respondents who left the question about their gender blank are not included.
24 April 2025 03:48:39
ries to share and implement strategies for improving diversity,
equity, and inclusion and thus decrease instances of harass-
ment and discrimination. The goal is a more respectful, wel-
coming, and inclusive community. The Effective Practices for
Physics Programs guide, which was created in a collaboration
between APS and the American Association of Physics Teachers,
also provides practices and strategies to improve physics-
department culture in many areas, including ethics. 4
3. Identify new ways to assess researchers
The San Francisco Declaration on Research Assessment5 and the
Leiden Manifesto 6 are important initiatives in the social sci-
ence and biology communities that promote moving beyond
simplistic metrics, such as journal impact factors or h- indexes,
to evaluate the quality of scientifi c work. APS should consider
following suit and establishing a task force to develop ideas for
assessing physics research quality that can guide hiring and
tenure or promotion review at research institutions.
4. Highlight accountability
To demonstrate that the APS Guidelines on Ethics are taken
seriously, APS should fi nd ways to highlight when its policy
for revocation of honors has been implemented and an honor
has been revoked. APS should also promote structural best
practices that reduce the absolute power that an individual
research adviser has over the careers of graduate students and
postdocs. For example, rather than relying solely on the opin-
ion of the adviser, a departmental commiĴee could meet once
a year or more to assess a student’s progress, identify problems
and roadblocks, and help ensure timely completion of the PhD.
5. Expand the concerned community
In an increasingly interdisciplinary scientific world, changing
the physics culture and advancing ethical best practices can
only be accomplished by working with other scientifi c and
engineering societies. APS leadership should reach out to other
science- based organizations and explore mutual interests, ac-
tivities, and potential opportunities.
