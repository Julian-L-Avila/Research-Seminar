\begin{frame}{\textbf{Example 1: Asymmetric Dirac Equation}}

  \onslide<1->{\textbf{Research Question:} Can modifying the Dirac equation to
    allow asymmetric dispersion relations for particles and antiparticles lead to
  a consistent quantum field theory?}

  \vspace{0.3cm}

  \onslide<2->{\textbf{Hypothesis:} Introducing asymmetry in the Dirac equation
    will result in a Lorentz-covariant, renormalizable quantum electrodynamics
    that is empirically equivalent to the standard model, while offering insights
  into matter-antimatter asymmetry.}

  \vspace{0.3cm}

  \onslide<3->{\emph{Source:} Rigolin, G. (2023). Asymmetric particle-antiparticle
    Dirac equation: second quantization. \textit{J. Phys. G: Nucl. Part. Phys.},
  50(12), 125005. \url{https://arxiv.org/abs/2208.12239}} \cite{Rigolin2023}

\end{frame}

\begin{frame}{\textbf{Example 2: Chiral Symmetry and Neutrino Masses}}

  \onslide<1->{\textbf{Research Question:} How does incorporating chiral symmetry
  into the Dirac equation affect our understanding of neutrino masses and dark
  matter candidates?}

  \vspace{0.3cm}

  \onslide<2->{\textbf{Hypothesis:} Applying chiral symmetry through the
  irreducible representations of the Poincaré group in the Dirac equation
  explains the small masses of neutrinos and predicts new massive particles as
  dark matter candidates.}

  \vspace{0.3cm}

  \onslide<3->{\emph{Source:} Watson, T.B., \& Musielak, Z.E. (2020). Chiral
  Symmetry in Dirac Equation and its Effects on Neutrino Masses and Dark Matter.
  \textit{Int. J. Mod. Phys. A}, 35(30), 2050189.
  \url{https://arxiv.org/abs/2009.03720}} \cite{Watson2021}

\end{frame}
