\section{Introduction}

% Original Slide 1: Understanding Acuerdo N° 012 - Split into two slides

\begin{frame}{Understanding Acuerdo N° 012: Definition \& Purpose}
  \begin{itemize}
    \item<1-> \textbf{What is it?} An official agreement from the \textbf{Academic Council of Universidad Distrital Francisco José de Caldas}.
      \pause % Alternative to \onslide for simple sequential reveal
    \item<2-> \textbf{Core Purpose:} To \textbf{regulate and standardize the process of "trabajo de grado" (degree work)} for both technology and undergraduate programs.
  \end{itemize}
\end{frame}

  \begin{frame}{Understanding Acuerdo N° 012: Rationale \& Legal Basis}
    \begin{itemize}
      \item<1-> \textbf{Why was it created?}
        \begin{itemize}
          \item<2-> To provide clear guidelines for obtaining a university degree.
          \item<3-> It \textbf{modifies and updates previous agreements} (Acuerdo 031 of 2014 and Acuerdo 038 of 2015).
          \item<4-> Based on a \textbf{critical analysis of current training processes} and faculty experiences.
        \end{itemize}
        \pause
      \item<5-> \textbf{Legal Foundation:} Rooted in the \textbf{Estatuto Estudiantil (Acuerdo N° 027 of 1993, Article 70)}, which mandates degree work and its regulation by the Academic Council.
    \end{itemize}
  \end{frame}

% Original Slide 2: Degree Work: Purpose and Prerequisites - Split into two slides

  \begin{frame}{Degree Work: Defining its Purpose}
    \begin{itemize}
      \item<1-> \textbf{What is Degree Work?}
        \begin{itemize}
          \item<2-> A crucial \textbf{formative process} designed for degree attainment.
          \item<3-> Aims to enhance \textbf{holistic student development}, broadening horizons in:
            \begin{itemize}
              \item<4-> Research
              \item<5-> Creation \& Innovation
              \item<6-> Technology
              \item<7-> Social Projection
              \item<8-> Professional Preparation
            \end{itemize}
        \end{itemize}
    \end{itemize}
  \end{frame}

    \begin{frame}{Degree Work: Prerequisites \& Key Regulations}
      \begin{itemize}
        \item<1-> \textbf{When Can Students Start?}
          \begin{itemize}
            \item<2-> After successfully \textbf{approving at least 70\% of their academic credits}.
          \end{itemize}
          \pause
        \item<3-> \textbf{Key Regulations:}
          \begin{itemize}
            \item<4-> \textbf{Single Modality:} Students can only pursue one degree work modality at a time.
            \item<5-> \textbf{Modality Change:} A maximum of \textbf{one change of modality} is permitted.
            \item<6-> \textbf{Intellectual Property:} All degree work must adhere to national intellectual property laws and university statutes.
          \end{itemize}
      \end{itemize}
    \end{frame}

% Original Slide 3: The Degree Work Journey: Key Steps - Using \onslide for steps

    \begin{frame}{The Degree Work Journey: Key Steps}
      \begin{enumerate}
        \item<1-> \textbf{Proposal Submission:}
          \begin{itemize}
            \item<2-> Students submit their degree work request to the \textbf{Curricular Council} for approval.
            \item<3-> This also authorizes enrollment in the relevant academic spaces.
          \end{itemize}
          \onslide<4->
        \item<4-> \textbf{Enrollment:}
          \begin{itemize}
            \item<5-> Enrollment in "Trabajo de Grado" academic spaces occurs during the designated week in the academic calendar.
          \end{itemize}
          \onslide<6->
        \item<6-> \textbf{Development \& Guidance:}
          \begin{itemize}
            \item<7-> Students undertake their degree work autonomously, but under the \textbf{guidance of a Faculty Director}.
          \end{itemize}
          \onslide<8->
        \item<8-> \textbf{Final Submission \& Evaluation:}
          \begin{itemize}
            \item<9-> Upon completion, the final work is submitted for evaluation.
            \item<10-> Final grades are officially reported by the \textbf{Curricular Project Coordination} in the Academic Management System.
          \end{itemize}
      \end{enumerate}
    \end{frame}

    \section{Modalities of Degree Work}

% Original Slide 4: Exploring Degree Work Modalities - Split for better readability

    \begin{frame}{Exploring Degree Work Modalities: Overview}
      \begin{itemize}
        \item<1-> Acuerdo N° 012 outlines various modalities for degree work.
        \item<2-> Here are the primary options available:
          \begin{itemize}
            \item<3-> \textbf{Research, Research-Creation, Innovation}
            \item<4-> \textbf{Monograph}
            \item<5-> \textbf{Creation, Direction, or Interpretation}
            \item<6-> \textbf{Production of Academic Article}
        % Add a \pause or new slide if this list gets too long on one reveal
          \end{itemize}
      \end{itemize}
    \end{frame}

    \begin{frame}{Exploring Degree Work Modalities: Continued}
      \begin{itemize}
        \item[] Primary options available (continued):
          \begin{itemize}
            \item<1-> \textbf{Entrepreneurship Project}
            \item<2-> \textbf{Internship (Pasantía)}
            \item<3-> \textbf{Postgraduate Academic Spaces} (Professional)
            \item<4-> \textbf{Deepening Academic Spaces} (Technology)
          \end{itemize}
      \end{itemize}
      \begin{flushleft}
        \footnotesize{\onslide<5-> \textit{The following slides will delve into each of these modalities.}}
      \end{flushleft}
    \end{frame}

% Slides for each modality - using \onslide for sub-points

    \begin{frame}{Modality: Research, Research-Creation, Innovation}
      \begin{itemize}
        \item<1-> \textbf{Objective:} To cultivate and strengthen \textbf{investigative training}.
          \onslide<2->
        \item<2-> \textbf{Key Requirement:}
          \begin{itemize}
            \item<3-> Must be \textbf{linked to an institutionalized research structure} (e.g., institute, center, group, or seedbed) recognized by Minciencias.
            \item<4-> Requires an \textbf{"aval" (endorsement)} from the research structure and a faculty director.
          \end{itemize}
          \onslide<5->
        \item<5-> \textbf{Typical Duration:} One or two academic periods.
          \onslide<6->
        \item<6-> \textbf{Evaluation:} Final grade is an \textbf{arithmetic average} of the director's and evaluator's grades. \textbf{Socialization} is a key component of the final grade.
      \end{itemize}
    \end{frame}

    \begin{frame}{Modality: Monograph}
      \begin{itemize}
        \item<1-> \textbf{Objective:} To produce an \textbf{original, argumentative academic work} that analyzes a specific topic.
          \onslide<2->
        \item<2-> \textbf{Methodology:} Primarily relies on the analysis of \textbf{primary and secondary sources}.
          \onslide<3->
        \item<3-> \textbf{Typical Duration:} One or two academic periods.
          \onslide<4->
        \item<4-> \textbf{Evaluation:} Final grade is an \textbf{arithmetic average} of the director's and evaluator's grades. \textbf{Socialization} is included in the final grade.
      \end{itemize}
    \end{frame}

    \begin{frame}{Modality: Creation, Direction, or Interpretation}
      \begin{itemize}
        \item<1-> \textbf{Objective:} To engage with elements inherent to art or interdisciplinary fields with art, promoting \textbf{creation, analysis, direction, or execution of artistic works}.
          \onslide<2->
        \item<2-> \textbf{Format Flexibility:}
          \begin{itemize}
            \item<3-> Can be \textbf{individual or collective}.
            \item<4-> Group sizes approved by the Curricular Council (especially for arts programs).
            \item<5-> Individual reports are required for larger groups.
          \end{itemize}
          \onslide<6->
        \item<6-> \textbf{Typical Duration:} One or two academic periods.
          \onslide<7->
        \item<7-> \textbf{Evaluation:} Based on a \textbf{written report} demonstrating objective fulfillment. Final grade is an \textbf{arithmetic average} of the director's and evaluator's grades, with \textbf{socialization} included.
      \end{itemize}
    \end{frame}

    \begin{frame}{Modality: Production of Academic Article}
      \begin{itemize}
        \item<1-> \textbf{Objective:} The \textbf{publication of a scientific, technological, artistic, review, reflection, or case report article} in an indexed journal.
          \onslide<2->
        \item<2-> \textbf{Crucial Requirements:}
          \begin{itemize}
            \item<3-> Student(s) must be the \textbf{first author(s)}.
            \item<4-> Requires an \textbf{acceptance letter and final draft/proof} from the journal.
            \item<5-> Journal's indexation must be \textbf{verified by the Curricular Project Coordinator}.
            \item<6-> University affiliation must be \textbf{credited} in the publication.
          \end{itemize}
          \onslide<7->
        \item<7-> \textbf{Typical Duration:} One or two academic periods.
          \onslide<8->
        \item<8-> \textbf{Evaluation:} The \textbf{faculty director assigns the final grade}. \textbf{Socialization is required} but does not carry a specific grade; it's a prerequisite for evaluation.
      \end{itemize}
    \end{frame}

    \begin{frame}{Modality: Entrepreneurship Project}
      \begin{itemize}
        \item<1-> \textbf{Objective:} To develop a \textbf{business model or plan} that effectively addresses a social need.
          \onslide<2->
        \item<2-> \textbf{Typical Duration:} One or two academic periods.
          \onslide<3->
        \item<3-> \textbf{Evaluation:} Faculties define specific evaluation criteria. The final grade is an \textbf{arithmetic average} of the director's and evaluator's grades, with \textbf{socialization} included.
      \end{itemize}
    \end{frame}

% Original Slide 10: Internship (Pasantía) - Split for clarity of Duration/Types and other details

      \begin{frame}{Modality: Internship (Pasantía) - Objective \& Duration}
        \begin{itemize}
          \item<1-> \textbf{Objective:} To gain \textbf{theoretical-practical experience} through a stay in a legally constituted organization or university unit (social, cultural, artistic, or business).
            \onslide<2->
          \item<2-> \textbf{Duration:}
            \begin{itemize}
              \item<3-> Minimum \textbf{192 hours} for technology programs.
              \item<4-> Minimum \textbf{384 hours} for professional programs.
              \item<5-> Maximum of \textbf{two semesters}.
            \end{itemize}
        \end{itemize}
      \end{frame}

        \begin{frame}{Modality: Internship (Pasantía) - Types \& Requirements}
          \begin{itemize}
            \item<1-> \textbf{Types:}
              \begin{itemize}
                \item<2-> \textbf{External:} With an external organization, requiring an agreement and an appointed professional for accompaniment.
                \item<3-> \textbf{Internal:} Within a university unit, requiring an acceptance letter and an appointed university professional.
              \end{itemize}
              \onslide<4->
            \item<4-> \textbf{Mandatory:} \textbf{ARL (labor risk administrator) affiliation} is required before starting. The university handles ARL for internal internships.
              \onslide<5->
            \item<5-> \textbf{Evaluation:} Final grade is the \textbf{arithmetic average} of the director's and external codirector's (or invited director's) grades. \textbf{Socialization} is included.
          \end{itemize}
        \end{frame}

% Original Slide 11: Academic Spaces - Split for Professional and Technology Levels

        \begin{frame}{Modality: Postgraduate Academic Spaces (Professional Level)}
          \begin{itemize}
            \item<1-> \textbf{Objective:} Approving a set of academic spaces from a \textbf{university postgraduate program} relevant to the student's profile.
              \onslide<2->
            \item<2-> \textbf{Requirements:}
              \begin{itemize}
                \item<3-> Professional-level student.
                \item<4-> Written request and justification demonstrating relevance and affinity with the undergraduate program.
              \end{itemize}
              \onslide<5->
            \item<5-> \textbf{Duration:} One academic period (involves "Trabajo de Grado I" and "Trabajo de Grado II" in the same period).
              \onslide<6->
            \item<6-> \textbf{Evaluation:}
              \begin{itemize}
                \item<7-> Grades reported by the postgraduate program.
                \item<8-> Final grade is a \textbf{weighted average} of each space's grade by its credits.
                \item<9-> \textbf{Minimum overall final grade is 3.5}. All enrolled spaces must be approved.
              \end{itemize}
              \onslide<10->
            \item<10-> \textit{Note: Does not require a director or evaluator. Socialization is considered in the final note.}
          \end{itemize}
        \end{frame}

        \begin{frame}{Modality: Deepening Academic Spaces (Technology Level)}
          \begin{itemize}
            \item<1-> \textbf{Objective:} Approving obligatory and elective academic spaces from \textbf{any university professional-level program}.
              \onslide<2->
            \item<2-> \textbf{Requirements:}
              \begin{itemize}
                \item<3-> Technology-level student.
                \item<4-> Written request.
              \end{itemize}
              \onslide<5->
            \item<5-> \textbf{Duration:} One academic period.
              \onslide<6->
            \item<6-> \textbf{Evaluation:}
              \begin{itemize}
                \item<7-> Grades reported by the professional program.
                \item<8-> Final grade is a \textbf{weighted average} of each space's grade by its credits.
                \item<9-> All enrolled spaces must be approved.
              \end{itemize}
              \onslide<10->
            \item<10-> \textit{Note: Does not require a director or evaluator. Socialization is considered in the final note.}
          \end{itemize}
        \end{frame}

        \section{Key Aspects}

% Original Slide 12: Key Roles and Responsibilities - Lots of roles, good for \onslide

        \begin{frame}{Key Roles and Responsibilities (1/2)}
          \begin{itemize}
            \item<1-> \textbf{Academic Council:} Overarching authority for regulating and modifying degree work.
              \onslide<2->
            \item<2-> \textbf{Curricular Council:} Approves requests, designates directors/evaluators, grants extensions, and defines evaluation formats.
              \onslide<3->
            \item<3-> \textbf{Curricular Project Coordination:} Manages inscriptions, reports grades, sets deadlines, and verifies journal indexation.
              \onslide<4->
            \item<4-> \textbf{Student:} Responsible for developing the degree work autonomously, presenting proposals, and performing public socialization.
          \end{itemize}
        \end{frame}

        \begin{frame}{Key Roles and Responsibilities (2/2)}
          \begin{itemize}
            \item<1-> \textbf{Faculty Director:} Guides the student, monitors development, reports final grades, and ensures intellectual property compliance. Must be a career or special contract faculty.
              \onslide<2->
            \item<2-> \textbf{Faculty Evaluator:} Reviews the final work, attends socialization, and reports the final grade. Must be a career or special contract faculty.
              \onslide<3->
            \item<3-> \textbf{Academic Vice-Rectory:} Socializes proposals and manages ARL affiliation for internal internships.
          \end{itemize}
        \end{frame}

% Original Slide 13: Evaluation and Grading - Good for \onslide

          \begin{frame}{Evaluation and Grading: Process \& Passing Grade}
            \begin{itemize}
              \item<1-> \textbf{Evaluation Process:} A holistic assessment considering:
                \begin{itemize}
                  \item<2-> Fulfillment of objectives
                  \item<3-> Execution time and data analysis
                  \item<4-> Results, conclusions, and information management
                  \item<5-> Bibliography, commitment, and presentation
                \end{itemize}
                \onslide<6->
              \item<6-> \textbf{Minimum Passing Grade:} \textbf{3.0}
            \end{itemize}
          \end{frame}

          \begin{frame}{Evaluation and Grading: Final Grade Calculation}
            \begin{itemize}
              \item<1-> \textbf{Final Grade Calculation:}
                \begin{itemize}
                  \item<2-> \textbf{Most Modalities:} Arithmetic average of Director's and Evaluator's grades.
                  \item<3-> \textbf{Academic Article Production:} Director's grade only.
                  \item<4-> \textbf{Postgraduate \& Deepening Spaces:} Weighted average based on credits (with a minimum overall of 3.5 for Postgraduate).
                \end{itemize}
                \onslide<5->
              \item<5-> \textbf{Academic Calendar:} Standard academic calendar cuts \textbf{do not apply} to degree work academic spaces.
            \end{itemize}
          \end{frame}

% Original Slide 14: Duration and Extensions - Good for \onslide

          \begin{frame}{Duration and Extensions: Typical Duration}
            \begin{itemize}
              \item<1-> \textbf{Typical Duration:}
                \begin{itemize}
                  \item<2-> \textbf{1 or 2 academic periods:} Monograph, Research/Creation/Innovation, Entrepreneurship, Academic Production.
                  \item<3-> \textbf{1 or maximum 2 academic periods:} Internship.
                  \item<4-> \textbf{1 academic period:} Postgraduate and Deepening Spaces.
                  \item<5-> \textit{(An academic period is typically 18 weeks.)}
                \end{itemize}
            \end{itemize}
          \end{frame}

          \begin{frame}{Duration and Extensions: Granting Extensions}
            \begin{itemize}
              \item<1-> \textbf{Extensions:}
                \begin{itemize}
                  \item<2-> A \textbf{one-semester extension} can be granted \textbf{exceptionally} by the Curricular Council with proper justification.
                  \item<3-> Extensions are subject to the student not having exhausted their matrícula renewals.
                \end{itemize}
            \end{itemize}
          \end{frame}


% Original Slide 15: Distinctions for Excellence - Split for clarity

            \begin{frame}{Distinctions for Excellence: Purpose \& Awarding}
              \begin{itemize}
                \item<1-> \textbf{Purpose:} To recognize \textbf{outstanding degree works} demonstrating originality, creativity, significant results, or invention.
                  \onslide<2->
                \item<2-> \textbf{Awarded by:} The \textbf{Faculty Council}, upon written request from directors and evaluators.
                  \onslide<3->
                \item<3-> \textbf{Minimum Grades for Distinction:}
                  \begin{itemize}
                    \item<4-> \textbf{Meritoria:} 4.5 or higher.
                    \item<5-> \textbf{Laureada:} 5.0.
                  \end{itemize}
              \end{itemize}
            \end{frame}

              \begin{frame}{Distinctions for Excellence: Exclusions \& Criteria}
                \begin{itemize}
                  \item<1-> \textbf{Modalities Excluded from Distinctions:}
                    \begin{itemize}
                      \item<2-> Internships
                      \item<3-> Postgraduate Academic Spaces
                      \item<4-> Deepening Academic Spaces
                      \item<5-> Entrepreneurship Projects
                    \end{itemize}
                    \onslide<6->
                  \item<6-> \textbf{Specific Criteria for Eligible Modalities (Examples):}
                    \begin{itemize}
                      \item<7-> \textbf{Monograph:} Meritoria (new knowledge), Laureada (significant contribution to knowledge frontiers).
                      \item<8-> \textbf{Research/Creation/Innovation:} Meritoria (new knowledge, social appropriation evidence), Laureada (evidence of intellectual property product).
                      \item<9-> \textbf{Academic Production:} Meritoria (published in indexed journal Category B, Q3), Laureada (published in indexed journal Category A1, A2, Q1, Q2).
                    \end{itemize}
                \end{itemize}
              \end{frame}

              \section{Conclusion}

              \begin{frame}{Final Provisions}
                \begin{itemize}
                  \item<1-> \textbf{Effectiveness:} The agreement became effective the day after its communication or publication.
                    \onslide<2->
                  \item<2-> \textbf{Repeals:} This agreement \textbf{replaces and invalidates} previous agreements, specifically N° 031 of 2014 and N° 038 of 2015.
                    \onslide<3->
                  \item<3-> \textbf{No Recourse:} This agreement is final; no appeal is possible against its provisions.
                    \onslide<4->
                  \item<4-> \textbf{Publicity:} The university ensures the agreement is publicly available on its website.
                \end{itemize}
              \end{frame}
