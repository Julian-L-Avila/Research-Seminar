\section{What are Predatory Journals?}

\begin{frame}
  \frametitle{Definition}
  \justifying
  \alert{Jeffrey Beall}, librarian at the University of Colorado, coined the term \alert{"predatory publishers"}:
  \bigskip

  \onslide<2->{\textbf{“Organizations that publish counterfeit journals to exploit the open-access model in which the author pays.”}}

  \onslide<3->{\begin{itemize}
      \item Dishonest, lacking transparency
      \item Focus on \alert{profit over quality}
      \item Deceptive practices
  \end{itemize}}
\end{frame}

\begin{frame}
  \frametitle{Key Characteristics}
  \onslide<1->{\begin{itemize}
      \item No rigorous peer-review
      \item Rapid, superficial editorial processes
      \item Fake editorial boards
      \item \alert{Misleading metrics and claims}
  \end{itemize}}

  \onslide<2->{\bigskip
  \textbf{Main goal:} Maximize fees with minimum effort}
\end{frame}

\section{Why You Should Avoid Them}

\begin{frame}
  \frametitle{Deceptive Invitations}
  \justifying
  \onslide<1->{Predatory publishers often lure authors via flattering, unsolicited emails.}
  \bigskip

  \onslide<2->{\textbf{Tactics include:}
    \begin{itemize}
      \item Over-the-top praise
      \item Guaranteed acceptance
      \item Fast peer-review and publication
  \end{itemize}}
\end{frame}

\begin{frame}
  \frametitle{Example: Solicitation Email}
  \justifying
  \small
  \begin{quote}
    "Estimado/a J. Avila-Martinez,

    Mi nombre es Tatiana Manastyrly y soy editora en Eliva Press. Revisé su lista de publicaciones recientes y es impresionante. Estamos interesados en publicar un libro de sus trabajos, incluyendo este: \textit{"Slab Jet estudio numérico usando el marco de la RRMHD"}.

    Nos especializamos en la publicación y difusión de libros académicos. La publicación es gratuita para nuestros autores. Conservará los derechos de autor y ganará hasta un 50\% de regalías. Además, el libro se distribuirá a través de los canales de distribución de Amazon y también otros.

    ¿Quiere saber más sobre nuestros servicios y acuerdo de publicación?

    Su respuesta sería muy apreciada.

    --\\
    Saludos cordiales,\\
    Tatiana Manastyrly\\
    Editora"
  \end{quote}
\end{frame}

\begin{frame}
  \frametitle{Risks of Engaging}
  \onslide<1->{\begin{itemize}
      \item \alert{Fake peer review} undermines scientific discourse
      \item Work may be \alert{invisible or inaccessible}
      \item Publishers may \alert{vanish or never publish}
      \item No formal copyright or contracts may exist
  \end{itemize}}
\end{frame}

\section{How to Protect Yourself}

\begin{frame}
  \frametitle{Tools and Strategies}
  \onslide<1->{\textbf{To avoid predatory journals, you can:}}

  \bigskip
  \begin{itemize}
    \item \onslide<2->{\alert{Review journal archives}: off-topic or low-quality content?}
    \item \onslide<3->{\alert{Check website transparency}: peer-review process, APCs}
    \item \onslide<4->{\alert{Scrutinize emails}: look for typos, vague claims}
    \item \onslide<5->{\alert{Use tools like Think. Check. Submit. or DOAJ}}
  \end{itemize}
\end{frame}

\begin{frame}
  \frametitle{Final Thoughts}
  \onslide<1->{\bfseries Vigilance is key.}
  \bigskip

  \onslide<2->{\begin{itemize}
      \item Prioritize reputable publishers
      \item Seek guidance from experienced researchers or librarians
      \item \alert{Trustworthy science demands rigorous channels}
  \end{itemize}}
\end{frame}
