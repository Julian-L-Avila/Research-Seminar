\begin{frame}{Developing a Research Question}
\begin{block}{}
\onslide<+-> A RQ can take different formats depending on what you evaluate:

\vspace{0.3cm}

\onslide<+-> \textbf{Existence}: Confirm a phenomenon or rule out rival explanations.

\vspace{0.3cm}

\onslide<+-> \textbf{Description and Classification}: State uniqueness or categorize features.

\vspace{0.3cm}

\onslide<+-> \textbf{Composition}: Break down a whole into its components.

\vspace{0.3cm}

\onslide<+-> \textbf{Relationship}: Evaluate connections between variables.

\vspace{0.3cm}

\onslide<+-> \textbf{Causality}: Determine cause-effect links.
\end{block}
\end{frame}

\begin{frame}{Steps to Develop a RQ}
\begin{block}{}
\onslide<+-> Identify a broader topic of interest (e.g., tau neutrinos).

\vspace{0.3cm}

\onslide<+-> Do preliminary research: What is known? What is missing?
\end{block}
\end{frame}

\begin{frame}{Narrowing the Focus}
\begin{block}{}
\onslide<+-> Identify implied questions arising from gaps or needs.

\vspace{0.3cm}

\onslide<+-> Narrow the scope to focus the investigation.
\end{block}
\end{frame}

\begin{frame}{Evaluating the RQ}
\begin{block}{}
\begin{itemize}
\onslide<+-> \item Is it \textbf{clear}?
\onslide<+-> \item Is it \textbf{focused}?
\onslide<+-> \item Is it \textbf{complex}?
\onslide<+-> \item Is it \textbf{interesting} and \textbf{useful}?
\onslide<+-> \item Is it \textbf{researchable} and \textbf{measurable}?
\onslide<+-> \item Is it \textbf{appropriately scoped}?
\end{itemize}
\end{block}
\end{frame}

\begin{frame}{Creating Hypotheses}
\begin{block}{}
\onslide<+-> After formulating a RQ, think about potential directions for the research.

\vspace{0.3cm}

\onslide<+-> Consider arguments to support or refute based on possible outcomes.
\end{block}
\end{frame}

\begin{frame}{Understanding Implications}
\begin{block}{}
\onslide<+-> Reflect on how your research fills a gap in knowledge.

\vspace{0.3cm}

\onslide<+-> Assess the potential practical applications of the results.
\end{block}
\end{frame}

\begin{frame}{Writing Down the RQ}
\begin{block}{}
\onslide<+-> Divide the RQ into two or three main concepts.

\vspace{0.3cm}

\onslide<+-> Focus on keeping it simple, clear, and manageable.
\end{block}
\end{frame}
