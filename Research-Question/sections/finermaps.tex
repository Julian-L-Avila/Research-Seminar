\begin{frame}{Designing Good Research Questions}
  \begin{block}{\textbf{FINERMAPS Structure}}
    \begin{itemize}
      \onslide<+-> \item \textbf{F}easible
      \onslide<+-> \item \textbf{I}nteresting
      \onslide<+-> \item \textbf{N}ovel
      \onslide<+-> \item \textbf{E}thical
      \onslide<+-> \item \textbf{R}elevant
      \onslide<+-> \item \textbf{M}anageable
      \onslide<+-> \item \textbf{A}ppropriate
      \onslide<+-> \item \textbf{P}otential Value and Publishability
      \onslide<+-> \item \textbf{S}ystematic
  \end{itemize}
\end{block}
\end{frame}

\begin{frame}{Feasible}
  \begin{block}{}
    \onslide<+-> \textbf{Feasibility} means the research is achievable by the
    investigator.

    \vspace{0.3cm}

    \onslide<+-> Be realistic about the \textbf{scope} and \textbf{scale} of the
    project.
    Document unexpected problems and reflect on them.
  \end{block}
\end{frame}

\begin{frame}{Interesting}
  \begin{block}{}
    \onslide<+-> A \textbf{genuine interest} sustains motivation throughout the
    research.

    \vspace{0.3cm}

    \onslide<+-> Interest must be backed by \textbf{academic} and
    \textbf{intellectual} engagement.
  \end{block}
\end{frame}

\begin{frame}{Novel}
  \begin{block}{}
    \onslide<+-> A \textbf{novel} RQ fills a gap instead of repeating existing
    studies.

    \vspace{0.3cm}

    \onslide<+-> \textbf{Simplicity and clarity} strengthen the research
    direction and avoid confusion.
  \end{block}
\end{frame}

\begin{frame}{Ethical}
  \begin{block}{}
    \onslide<+-> \textbf{Ethics} is essential to protect participants and ensure
    integrity.

    \vspace{0.3cm}

    \onslide<+-> Obtain \textbf{clearances} from relevant authorities before
    beginning.
  \end{block}
\end{frame}

\begin{frame}{Relevant}
  \begin{block}{}
    \onslide<+-> The RQ must be \textbf{relevant} to the academic and
    professional field.

    \onslide<+-> It may:
    \begin{itemize}
      \onslide<+-> \item Fill knowledge gaps
      \onslide<+-> \item Analyse assumptions
      \onslide<+-> \item Monitor developments
      \onslide<+-> \item Compare approaches
      \onslide<+-> \item Test theories
  \end{itemize}
\end{block}
\end{frame}

\begin{frame}{Manageable and Appropriate}
  \begin{block}{}
    \onslide<+-> \textbf{Manageable}: The research should be within the
    researcher's capacity.

    \vspace{0.3cm}

    \onslide<+-> \textbf{Appropriate}: The RQ should fit logically and
    scientifically into the community and institution.
  \end{block}
\end{frame}

\begin{frame}{Potential Value and Publishability}
  \begin{block}{}
    \onslide<+-> A strong RQ offers \textbf{potential value} to its field.

    \vspace{0.3cm}

    \onslide<+-> It increases the chances of \textbf{publication} and broader
    impact.
  \end{block}
\end{frame}

\begin{frame}{Systematic}
  \begin{block}{}
    \onslide<+-> Research should follow a \textbf{structured sequence of steps}.

    \vspace{0.3cm}

    \onslide<+-> \textbf{Creative thinking} is encouraged within the framework.
  \end{block}
\end{frame}

