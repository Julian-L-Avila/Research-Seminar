\begin{frame}{From Research Question to Hypothesis}
  \begin{block}{}
    \onslide<+-> A good RQ requires a \textbf{thorough literature review} and
    \textbf{deep insight} into the specific area or problem.

    \vspace{0.3cm}

    \onslide<+-> The RQ must be \textbf{focused} and \textbf{simple}.

  \end{block}
\end{frame}

\begin{frame}{Example}
  \begin{block}{}

    \onslide<+-> For example, Dirac posed the RQ:
    \textit{What is a relativistic wave equation for the electron compatible
    with quantum mechanics?}

    \vspace{0.3cm}

    \onslide<+-> Initial attempts had led to the Klein-Gordon equation:
    \[
      \left( \eta_{\mu\nu} \partial_\mu \partial_\nu + \frac{m^2 c^2}{\hbar^2} \right) \phi = 0
    \]

  \end{block}
\end{frame}

\begin{frame}{Example}
  \begin{block}{}

    \onslide<+-> Dirac hypothesized that \textbf{spin is intrinsic} to a
    relativistic wave equation, leading to a first-order formulation using Clifford algebra:
    \[
      \left( i \hbar \slashed{\partial} - m c \right) \psi = 0
      \quad \text{with} \quad
      \slashed{\partial} = \gamma_\mu \partial_\mu
    \]
    where \( \gamma_\mu \) are the generators of the Clifford algebra \(\text{Cl}(1,3)\),
    satisfying
    \[
      \gamma_\mu \gamma_\nu + \gamma_\nu \gamma_\mu = 2 \eta_{\mu\nu}
    \]
    and \(\eta_{\mu\nu} = \mathrm{diag}(1, -1, -1, -1)\).
    Also \( \sigma_i = \gamma_i \gamma_0 \), this implies intrinsic spin.

    \vspace{0.3cm}

    \onslide<+-> This illustrates how a precise RQ, identifying gaps in current
    models, can drive groundbreaking hypotheses.
  \end{block}
\end{frame}

