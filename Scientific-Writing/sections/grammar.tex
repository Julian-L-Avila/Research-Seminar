\begin{frame}{Using Tenses Consistently}
  Different tenses serve different purposes: \pause
  \begin{itemize}
    \item \textbf{Present tense}: Established facts, research questions, and beliefs at the time of the study. \pause
    \item \textbf{Past tense}: Describing methods, results, and attributing previous work that is not yet established fact.
  \end{itemize}

  Avoid switching tenses within the same sentence or adjacent sentences unless necessary.
\end{frame}

\begin{frame}{Adjectives and Adverbs}
  \textbf{Adjectives}
  \begin{itemize}
    \item Use strong adjectives when justified: \textit{urgent}, \textit{dangerous}, \textit{essential}.
    \item Avoid weak or unnecessary adjectives: \textit{particular}, \textit{apparent}, \textit{notable}.
  \end{itemize}
  \pause
  \textbf{Adverbs}
  \begin{itemize}
    \item Use adverbs only when they add meaning.
    \item Example: \textit{Up to 85\% of students mistakenly believe they are
      impostors and not intelligent enough to present their research at a
      conference.} (\textit{mistakenly} is essential to the meaning)
  \end{itemize}
\end{frame}

\begin{frame}{Constructing a Strong Paragraph}
  A \textbf{paragraph} is not just a collection of sentences.
  \pause

  A well-constructed paragraph should:
  \begin{itemize}
    \item Contain a group of related ideas. \pause
    \item Follow a logical order (e.g., most important to least important, earliest to latest). \pause
    \item Have smooth transitions between sentences. \pause
    \item Start with a clear topic sentence. \pause
    \item End with a concluding sentence.
  \end{itemize}
\end{frame}

\begin{frame}{Example: Before Revision}
  \textbf{Before:}
  \begin{quote}
    The impact of screen time on psychological health is controversial.
    Smartphone use in younger people has consistently increased in recent years.
    Controversy always arises around the appropriate use of new disruptive
    technology.
    The arguments often collapse into scaremongering claims.
    We remain influenced by correlational findings.
    The confusion continues.
    We need to critically appraise current research.
    We need to identify the key questions.
    We need to determine what research is needed to answer these questions.
  \end{quote}
\end{frame}

\begin{frame}{Example: After Revision (English)}
  \textbf{After:}
  \begin{quote}
    The impact of screen time on psychological health is controversial.
    \textbf{In recent years}, smartphone use among younger people has
    consistently increased.
    Controversy always arises regarding the appropriate use of new disruptive
    technology.
    \textbf{However}, arguments often collapse into scaremongering claims,
    \textbf{and} we remain influenced by correlational findings.
    \textbf{Consequently}, confusion persists.
    \textbf{To advance}, we must critically appraise current research, identify
    key questions, \textbf{and} determine what studies are needed to answer them.
  \end{quote}
\end{frame}

\begin{frame}{Ejemplo: Antes de la Revisión (Español)}
  \textbf{Antes:}
  \begin{quote}
    El impacto del tiempo frente a la pantalla en la salud psicológica es
    controvertido.
    El uso de teléfonos inteligentes entre los jóvenes ha aumentado
    constantemente en los últimos años.
    Siempre surge controversia sobre el uso apropiado de nuevas tecnologías disruptivas.
    Los argumentos a menudo se reducen a afirmaciones alarmistas.
    Seguimos influenciados por hallazgos correlacionales.
    La confusión continúa.
    Necesitamos evaluar críticamente la investigación actual.
    Necesitamos identificar las preguntas clave.
    Necesitamos determinar qué investigaciones se necesitan para responderlas.
  \end{quote}
\end{frame}

\begin{frame}{Ejemplo: Después de la Revisión (Español)}
  \textbf{Después:}
  \begin{quote}
    El impacto del tiempo frente a la pantalla en la salud psicológica es
    controvertido.
    \textbf{En los últimos años}, el uso de teléfonos inteligentes entre los
    jóvenes ha aumentado constantemente.
    Siempre surge controversia sobre el uso apropiado de nuevas tecnologías
    disruptivas.
    \textbf{Sin embargo}, los argumentos a menudo se reducen a afirmaciones
    alarmistas, \textbf{y} seguimos influenciados por hallazgos correlacionales.
    \textbf{En consecuencia}, la confusión persiste.
    \textbf{Para avanzar}, debemos evaluar críticamente la investigación actual,
    identificar las preguntas clave, \textbf{y} determinar qué estudios se
    necesitan para responderlas.
  \end{quote}
\end{frame}
