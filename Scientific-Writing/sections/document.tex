\begin{frame}{Planning the Manuscript}
  Avoid staring at a blank page—begin with structured planning.
  \pause
  \begin{itemize}
    \item Write the \textbf{Methods} section first (often the easiest part).
      \pause
    \item Prepare key figures and tables before writing results.
      \pause
    \item Summarize findings in short sentences to guide the message.
      \pause
    \item Draft a \textbf{problem statement} before tackling introduction and discussion.
  \end{itemize}
\end{frame}

\begin{frame}{Using a Mindmap}
  Create an electronic \textbf{mindmap} early in your research.
  \pause
  \begin{itemize}
    \item Include four sections: Introduction, Methods, Results, Discussion.
      \pause
    \item Continuously update it as ideas evolve.
      \pause
    \item Write ideas in sentences (or include verbs) for clarity.
      \pause
    \item Attach reference manager citations to streamline the writing process.
  \end{itemize}
\end{frame}

\begin{frame}{Writing the Introduction}
  \textbf{Goal}: Explain why the study is important and worth reading.
  \vspace{0.5cm}
  \pause
  \begin{columns}
    \column{0.5\textwidth}
    \begin{itemize}
      \item Provide only relevant background information.
        \pause
      \item Clearly define the research problem.
        \pause
      \item Present a logical rationale:
    \end{itemize}
    \pause
    \column{0.5\textwidth}
    \begin{itemize}
      \item Context (what is known)
      \item Problem (what is unknown)
      \item Research question/hypothesis
      \item Approach to answering/testing it
    \end{itemize}
  \end{columns}
\end{frame}

\begin{frame}{Writing the Methods Section}
  The \textbf{methods} section should be detailed yet concise.
  \pause

  \textbf{Do's}
  \begin{itemize}
    \item Selection and source of materials/participants.
    \item Study design (temperature, time, dose, etc.).
    \item Outcome measures.
    \item Statistical techniques (randomization, p-values, etc.).
    \item Ethics approval, if required.
  \end{itemize}
  \pause
  Ensure the study is reproducible.
\end{frame}

\begin{frame}{Presenting Results}
  Present key findings in a logical sequence.
  \vspace{0.5cm}
  \pause

  \begin{columns}
    \column{0.5\textwidth}
    \textbf{Do's}
    \begin{itemize}
      \item Use “significant” only for statistical results.
      \item Use precise descriptions.
      \item Use figures/tables with clear legends to avoid redundancy.
    \end{itemize}
    \pause
    \column{0.5\textwidth}
    \alert{\textbf{Don'ts}}
    \begin{itemize}
      \item Repeating visual data in text.
      \item Drawing conclusions (save for discussion).
      \item Excessive use of references.
    \end{itemize}
  \end{columns}
\end{frame}

\begin{frame}{Structuring the Discussion}
  \begin{itemize}
    \item Summarize main findings in the first paragraph.
      \pause
    \item Discuss strengths and weaknesses of the study.
      \pause
    \item Compare with existing research.
      \pause
    \item Provide a balanced conclusion and future research directions.
  \end{itemize}
  \pause

  \vspace{1cm}
  \begin{columns}
    \column{0.5\textwidth}
    \textbf{Do's}
    \begin{itemize}
      \item Keep arguments structured.
      \item Base conclusions on results.
    \end{itemize}
    \column{0.5\textwidth}
    \alert{\textbf{Don'ts}}
    \begin{itemize}
      \item Repetition of results.
      \item Unstructured arguments.
    \end{itemize}
  \end{columns}
\end{frame}

\begin{frame}{Writing the Abstract}
  Write the abstract last.
  \pause

  \vspace{0.5cm}
  \begin{columns}
    \column{0.5\textwidth}
    \textbf{Do's}
    \begin{itemize}
      \item Background (short and relevant).
      \item Methods (minimal details).
      \item Results (bulk of the abstract, ~50%).
      \item Conclusion (essential).
    \end{itemize}
    \pause
    \column{0.5\textwidth}
    \alert{\textbf{Don'ts}}
    \begin{itemize}
      \item Use vague questions.
      \item Miss data points.
      \item Allow “conclusion creep.”
    \end{itemize}
  \end{columns}

  \pause
  \vspace{0.5cm}

  Use concise sentences and active voice.
\end{frame}
