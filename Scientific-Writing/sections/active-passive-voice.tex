\begin{frame}{Active vs. Passive Voice: The Basics}
  \textbf{Active Voice} \pause
  \begin{itemize}
    \item The subject (actor) comes first and performs the action.
    \item The object (recipient) follows the verb.
    \item Example: \textbf{The team calculated the optimum pH.}
  \end{itemize}
  \textbf{Passive Voice} \pause
  \begin{itemize}
    \item The object (recipient) comes first.
    \item The subject (actor) appears later or is omitted.
    \item Additional words (\textit{is, was, are, being, by}) are required.
    \item Example: \textbf{The optimum pH was calculated by the team.}
  \end{itemize}
\end{frame}

\begin{frame}{Why Use Active Voice?}
  \begin{itemize}
    \item \textbf{Clearer and more concise.} \pause
    \item Highlights the subject—useful when it is as important as the object. \pause
    \item Preferred by many scientific journals.
  \end{itemize}
\end{frame}

\begin{frame}{Why Use Passive Voice?}
  \begin{itemize}
    \item \textbf{Sounds more formal}, leading to its traditional use in scientific writing. \pause
    \item Useful when the subject is unknown, obvious, or irrelevant.
    \item Emphasizes the object when it is more important than the subject.
  \end{itemize}
\end{frame}

\begin{frame}{Which Voice to Use?}
  \begin{itemize}
    \item Many journals prefer \textbf{active voice}—check author guidelines. \pause
    \item Active voice does not require a personal pronoun:
      \begin{itemize}
        \item \textbf{Process X improves yield.} \pause
      \end{itemize}
    \item Passive voice example:
      \begin{itemize}
        \item \textbf{Yield is improved by using Process X.}
      \end{itemize}
  \end{itemize}
\end{frame}

\begin{frame}{When to Use Active Voice}
  \begin{itemize}
    \item When the journal expects or prefers it. \pause
    \item To enhance readability and clarity. \pause
    \item When identifying the subject is important.
  \end{itemize}
\end{frame}

\begin{frame}{When to Use Passive Voice}
  \begin{itemize}
    \item Often preferred in methods sections. \pause
    \item When the actor is irrelevant:
      \begin{itemize}
        \item \textbf{These emissions are produced by diesel engines.}
      \end{itemize}
  \end{itemize}
\end{frame}

\begin{frame}{Using \textit{I} and \textit{We}}
  \begin{itemize}
    \item \textbf{We} is now widely accepted in scientific writing. \pause
    \item \textbf{I} is occasionally used in theses, especially for method justifications. \pause
    \item Some fields (e.g., physics) discourage first-person usage—always check journal guidelines.
  \end{itemize}
\end{frame}
