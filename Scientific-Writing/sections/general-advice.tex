\begin{frame}{The Goal of Scientific Writing}
  \begin{block}{Clarity Above All}
    \begin{center}
      \Large \itshape ``There is really only one essential goal in scientific writing: clarity."
    \end{center}
    \vspace{0.3cm}
    \hfill \textbf{– Robert Day} \cite{Dixon2020}
  \end{block}
\end{frame}

\begin{frame}{Useful Tips for Writing}
  \begin{itemize}
    \item Consider your audience: background, jargon, and interests. \pause
    \item Structure and clarity help readers navigate your writing. \pause
    \item Use simple, precise language to avoid unnecessary complexity.
  \end{itemize}
  \vspace{0.5cm}
  \alert{\textbf{These are not strict rules, but guidelines.}}
\end{frame}

\begin{frame}{Crafting Clear Sentences}
  \begin{itemize}
    \item Express one or two closely related ideas per sentence. \pause
    \item Emphasize the most important part at the beginning. \pause
    \item Keep subject and verb together for clarity. \pause
    \item Avoid excessive embedded clauses. \pause
    \item Sentences longer than 30 words should often be split.
  \end{itemize}
\end{frame}

\begin{frame}{Example: Improving Clarity}
  \textbf{Original:}
  \vspace{0.3cm}
  \begin{quote}
    Factors such as root depth, root density, water availability through
    different irrigation methods and more recently rhizosphere management affect
    rice crop hydration.
  \end{quote} \pause
  \vspace{0.5cm}
  \textbf{\textcolor{primary}{Revised:}}
  \begin{quote}
    Rice crop hydration is affected by factors such as root depth, root density,
    water availability through different irrigation methods, and rhizosphere
    management.
  \end{quote}
\end{frame}

\begin{frame}{Claridad en Español}
  \textbf{Original:}
  \begin{quote}
    Factores como la profundidad de las raíces, la densidad radicular, la disponibilidad de agua a través de diferentes métodos de riego y, más recientemente, la gestión de la rizosfera afectan la hidratación del cultivo de arroz.
  \end{quote} \pause
  \vspace{0.3cm}
  \textbf{\textcolor{primary}{Revisado:}}
  \begin{quote}
    La hidratación del cultivo de arroz se ve afectada por factores como la profundidad de las raíces, la densidad radicular, la disponibilidad de agua mediante distintos métodos de riego y la gestión de la rizosfera.
  \end{quote}
\end{frame}

